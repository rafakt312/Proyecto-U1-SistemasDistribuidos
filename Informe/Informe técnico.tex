\documentclass[a4paper,12pt]{article}
\usepackage[utf8]{inputenc}
\usepackage{geometry}
\geometry{margin=2.5cm}
\usepackage{setspace}
\usepackage{titling}

\begin{document}

% Portada
\begin{titlepage}
    \centering
    \vspace*{3cm}
    {\Huge\bfseries Proyecto 1 - Sistema distribuido \par}
    \vspace{2cm}
    {\Large Informe Técnico \par}
    \vspace{2cm}
    {\large Fecha: 7 de octubre de 2026 \par}
    \vspace{2cm}
    {\large
        \textbf{Integrantes:} \\
        Javier Gamboa \\
        Rafael Gonzalez \\
        Victor Cornejo
    }
    \vfill
\end{titlepage}

% Índice
\tableofcontents
\newpage

\section{Introducción}
La morfología matemática es una técnica de procesamiento de imágenes ampliamente utilizada
para la eliminación de ruido y la mejora de estructuras en fotografías digitales.
Entre sus operaciones fundamentales se encuentran la \textbf{erosión}, que tiende a eliminar
los detalles claros reduciendo regiones brillantes, y la \textbf{dilatación}, que expande
estas regiones y elimina detalles oscuros.

En este proyecto se implementó un sistema en \textbf{Java} que permite aplicar ambas
operaciones sobre imágenes en formato \textbf{PNG} y en modelo de color \textbf{RGB}.
La solución fue desarrollada en dos variantes: una \textbf{secuencial} y otra
\textbf{paralela}, con el objetivo de analizar las diferencias de rendimiento y validar
los resultados generados bajo distintos elementos estructurantes.

\section{Objetivos}
\begin{itemize}
    \item Implementar en Java las operaciones de erosión y dilatación en imágenes a color.
    \item Diseñar una solución \textbf{secuencial} que sirva como referencia para validar
    la corrección de la versión paralela.
    \item Desarrollar una implementación \textbf{paralela} mediante división de la imagen
    en submatrices (tiles) utilizando múltiples hilos.
    \item Comparar el rendimiento entre las versiones secuencial y paralela utilizando
    distintos tamaños de imagen y elementos estructurantes.
    \item Elaborar un informe técnico que documente el proceso, las decisiones de diseño
    y los resultados experimentales obtenidos.
\end{itemize}

\section{Desarrollo}
\subsection{Estructura del proyecto}
El proyecto se organizó siguiendo la convención de Maven, con los siguientes módulos
principales:
\begin{itemize}
    \item \textbf{core}: contiene las implementaciones de morfología secuencial y paralela,
    además del módulo de benchmark.
    \item \textbf{model}: define las estructuras de datos utilizadas como la política de borde,
    el tipo de operación y los elementos estructurantes.
    \item \textbf{util}: funciones auxiliares para la lectura/escritura de imágenes y medición de tiempos.
    \item \textbf{menu}: controlador de la aplicación, encargado de recibir parámetros de ejecución
    o interactuar con el usuario mediante consola.
\end{itemize}

\subsection{Implementación secuencial}
La versión secuencial recorre cada píxel de la imagen y aplica el elemento estructurante
sobre sus vecinos, reemplazando el valor del píxel central por el mínimo o máximo
dependiendo de la operación (erosión o dilatación).  
El manejo de bordes se realizó bajo dos políticas:
\begin{itemize}
    \item \textbf{Ignore}: se ignoran los vecinos que caen fuera de los límites de la imagen.
    \item \textbf{Pad}: se rellenan los valores fuera de rango con un valor neutro (255 en erosión,
    0 en dilatación).
\end{itemize}

\subsection{Implementación paralela}
Para la versión paralela se dividió la imagen en \textbf{submatrices horizontales} o tiles.
Cada tile incluye un \textbf{halo} adicional para evitar inconsistencias en los bordes
al aplicar el elemento estructurante.  
La paralelización se implementó mediante un \texttt{ExecutorService} en Java, asignando
a cada hilo el procesamiento de un bloque independiente.

Esta solución puede clasificarse en la \textbf{arquitectura MIMD} de Flynn, ya que
diferentes hilos ejecutan instrucciones distintas sobre distintos datos.

\subsection{Elementos estructurantes}
Se implementaron cinco elementos estructurantes:
\begin{enumerate}
    \item Cuadrado 3x3
    \item Cruz 3x3
    \item X 3x3
    \item Línea horizontal 1x3
    \item Diamante 5x5
\end{enumerate}

\subsection{Resultados experimentales}
Se evaluaron distintas imágenes con tamaños de hasta 10.000 x 10.000 píxeles.
Los experimentos incluyeron mediciones de tiempo para:
\begin{itemize}
    \item \textbf{Erosión vs. Dilatación}
    \item \textbf{Secuencial vs. Paralelo} con 2, 4, 8 y 16 hilos.
    \item Distintos elementos estructurantes.
\end{itemize}

Los tiempos fueron registrados únicamente durante la fase de cómputo, excluyendo
lectura y escritura de archivos. Para cada caso se realizaron tres ejecuciones,
almacenando el promedio y desviación estándar.

\section{Conclusiones}
El desarrollo de este proyecto permitió comprender en detalle cómo la
\textbf{paralelización} mejora el rendimiento en operaciones intensivas de procesamiento
de imágenes.  
Las principales conclusiones fueron:
\begin{itemize}
    \item La implementación paralela mostró mejoras significativas en imágenes grandes
    (superiores a 2000 x 2000 píxeles).
    \item El tiempo de ejecución decrece al aumentar los hilos, aunque la ganancia
    se estabiliza a partir de un número similar al de núcleos físicos de la CPU.
    \item La política de borde \textbf{Pad} asegura mayor consistencia visual en los resultados,
    mientras que \textbf{Ignore} es más eficiente en tiempo.
    \item El uso de un diseño modular permitió mantener la equivalencia entre los resultados
    de la versión secuencial y paralela.
\end{itemize}

\end{document}
